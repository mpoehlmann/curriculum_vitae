% !TEX TS-program = XeLaTeX
\documentclass[a4paper,10pt]{article}

%A Few Useful Packages
\usepackage{marvosym}
\usepackage{fontspec} 					%for loading fonts
\usepackage{xunicode,xltxtra,url,parskip} 	%other packages for formatting
\RequirePackage{color,graphicx}
\usepackage[usenames,dvipsnames]{xcolor}
\usepackage[big]{layaureo} 				%better formatting of the A4 page
% an alternative to Layaureo can be ** \usepackage{fullpage} **
\usepackage{supertabular} 				%for Grades
\usepackage{titlesec}					%custom \section

%My packages
\usepackage{array}
\usepackage{soul}
\usepackage{xspace}
\usepackage{cite}
\usepackage{bibentry}
\usepackage{colortbl}
\makeatletter\let\saved@bibitem\@bibitem\makeatother


%Setup hyperref package, and colours for links
\usepackage{hyperref}
\makeatletter\let\@bibitem\saved@bibitem\makeatother
\definecolor{linkcolour}{rgb}{0,0.2,0.6}
\hypersetup{colorlinks,breaklinks,urlcolor=linkcolour, linkcolor=linkcolour}

%FONTS
\defaultfontfeatures{Mapping=tex-text}
%\setmainfont[SmallCapsFont = Fontin SmallCaps]{Fontin}
%%% modified for Karol Kozioł for ShareLaTeX use
\setmainfont[SmallCapsFont = Fontin-SmallCaps.otf, BoldFont = Fontin-Bold.otf, ItalicFont = Fontin-Italic.otf]{Fontin.otf}
%%%

%CV Sections inspired by: 
%http://stefano.italians.nl/archives/26
\titleformat{\section}{\Large\scshape\raggedright}{}{0em}{}[\titlerule]
\titlespacing{\section}{0pt}{3pt}{3pt}
%Tweak a bit the margins

%\addtolength{\voffset}{-1.3cm}
\addtolength{\oddsidemargin}{-0.4375in} %-0.4375in
\addtolength{\evensidemargin}{-0.4375in} %-0.4375in
\addtolength{\textwidth}{0.875in} %0.875in
\addtolength{\topmargin}{-0.21875in} %-0.4375in
\addtolength{\textheight}{0.4375in} %0.875in

%-------------WATERMARK TEST [**not part of a CV**]---------------
\usepackage[absolute]{textpos}

\setlength{\TPHorizModule}{30mm}
\setlength{\TPVertModule}{\TPHorizModule}
\textblockorigin{2mm}{0.65\paperheight}
\setlength{\parindent}{0pt}


%--------------------CUSTOM COMMANDS----------------------
\newcommand{\PreserveBackslash}[1]{\let\temp=\\#1\let\\=\temp}
\newcolumntype{R}[1]{>{\PreserveBackslash \raggedleft}p{#1}}
%\newcolumntype{L}[1]{>{\PreserveBackslash \raggedright \hangindent=1em}p{#1}}
\newcolumntype{L}[1]{>{\PreserveBackslash \raggedright}p{#1}}
\newcolumntype{B}[1]{>{\PreserveBackslash \raggedright \hangindent=2em}p{#1}}

\newcommand{\UCD}{\textbf{University of California, Davis\xspace}}
\newcommand{\UMN}{\textbf{University of Minnesota\xspace}}

%--------------------------------------------------------
%--------------------BEGIN DOCUMENT----------------------
%--------------------------------------------------------
\begin{document}
\pagestyle{empty} % non-numbered pages
\font\fb=''[cmr10]'' %for use with \LaTeX command

\nobibliography{michael_poehlmann_pubs.bib}
\bibliographystyle{elsarticle-num} %unsrt

%--------------------TITLE-------------
\par{\centering
		{\Huge Michael Poehlmann   %\textsc{Poehlmann}
	}\bigskip\par}



%---------------------------------------------------------------
%--------------------SECTIONS-----------------------------------
%---------------------------------------------------------------
%%% Personal Information
\section{Personal Information}
\begin{tabular}{rl}
\textsc{Full Name:}             & David-Michael T. Poehlmann \\ %(\textit{formerly David Michael DeGuire}) \\
%\textsc{Previous Names:}   & David Michael DeGuire \\
\textsc{email:}                 & \href{mailto:poehlmann@ucdavis.edu}{poehlmann@ucdavis.edu} \\
\textsc{Website:}               & \href{https://mpoehlmann.github.io}{mpoehlmann.github.io}\\
\textsc{Phone:}                 & (608) 630-0431\\
\textsc{Address:}               & 1850 Hanover Dr. \#112, Davis, CA 95616 \\
% \multicolumn{2}{c}{} \\
\end{tabular}


%%%%%%%%%%%%%%%%%%%%%%%%%%%%%%%%%%%%%%%%%%%%%%%%%%
%%% Education
%%%%%%%%%%%%%%%%%%%%%%%%%%%%%%%%%%%%%%%%%%%%%%%%%%
\section{Education}
% \begin{tabular}{r|l}    %R{1.5cm} | L{14cm}} %r|l}
\begin{tabular}{R{1.5cm}|L{14cm}}
\emph{Present}  &  PhD candidate in \textsc{Physics}, \UCD, Davis, CA \\
\textsc{Sept 2018} & Degree expected in 2023 \\
& \textsc{Gpa}: 3.55/4 \\

\arrayrulecolor{white}\hline \hline \hline \arrayrulecolor{black}

\textsc{May 2018} & Honors BSc in \textsc{Physics}, \textit{cum laude}, \UMN, Minneapolis, MN \\
\textsc{Sept 2014} 	& \textsc{Gpa}: 3.60/4 \\ 
	% & Honors Thesis: ``Gadolinium-loaded Plastic Scintillator for Neutron Detection'' \\
	% & \hspace{5mm} \small Advisor: Prof. Priscilla Cushman \\

\end{tabular}


%%%%%%%%%%%%%%%%%%%%%%%%%%%%%%%%%%%%%%%%%%%%%%%%%%
%%% Research experiences
%%%%%%%%%%%%%%%%%%%%%%%%%%%%%%%%%%%%%%%%%%%%%%%%%%
\section{Research Experience}
\begin{tabular}{R{1.5cm}|L{14cm}}
\emph{Present}  & \textsc{DarkSide} Collaboration, \UCD \\
\textsc{Jul 2018} & \textit{PI: Prof. Emilija Pantic} \\
& {\footnotesize Worked on experimental design, data acquisition system development, hardware development, and detector commissioning for ARIS-ER. Ran Monte Carlo simulations in \textsc{Geant4} for the design of \textsc{DarkSide-20k}, with a focus on backgrounds and optical modeling. Contributed to the analysis of data collected by DarkSide-50.} \\

\arrayrulecolor{white}\hline \hline \hline \arrayrulecolor{black}

%\textsc{Jun 2018} \textsc{Jan 2016} & \textsc{Super Cryogenic Dark Matter Search (SuperCDMS)} Collaboration, \textbf{University of Minnesota} \\
\textsc{Jun 2018} & \textsc{Cryogenic Dark Matter Search (CDMS)} Collaboration, \UMN \\
\textsc{Jan 2016} & \textit{PI: Prof. Priscilla Cushman} \\
& {\footnotesize Developed components for the active neutron veto of the proposed \textsc{SuperCDMS SNOLAB} detector. Worked on loading gadolinium into plastic scintillator and characterized sample properties.} \\

\arrayrulecolor{white}\hline \hline \hline \arrayrulecolor{black}

\textsc{Jun 2017}  & \textsc{Light Dark Matter eXperiment (LDMX)} Collaboration, \UMN \\
\textsc{Jan 2017} & \textit{PI: Prof. Jeremiah Mans} \\
& {\footnotesize Measured the event discrimination efficiency of thin plastic scintillator sheets for the \textsc{LDMX} experiment.} \\

\arrayrulecolor{white}\hline \hline \hline \arrayrulecolor{black}

\textsc{Jun 2016} & \textsc{Greven} Research Group, \UMN \\
\textsc{Sept 2015} & \textit{PI: Prof. Martin Greven} \\
& {\footnotesize Grew and analyzed Hg1201 crystals to collect data on possible mechanisms behind high-temperature superconductivity. Tasks included crystal growth in conventional box furnaces, sample annealing, and measurements of susceptibility using a MPMS instrument.} \\
 
\end{tabular}


\section{Publications}
\begin{enumerate}
\item \bibentry{cdmsGdPS2018} \\
    {\footnotesize First-authored paper for which I fabricated and characterized scintillator samples.}
\end{enumerate}


\section{Presentations}
\begin{enumerate}
\item \bibentry{APSFarWestFall2020} \\
    {\footnotesize Talk given virtually at the 2020 APS Far West Section meeting.}
\item \bibentry{FNALNewPerspectives2020} \\
    {\footnotesize Talk given virtually at the Fermilab New Perspectives 2020 conference.}
\item \bibentry{ARISERpresentationAPS2019} \\
    {\footnotesize Poster presented at the 2019 APS Far West Section meeting at Stanford University.}
\end{enumerate}


\section{Honors and Fellowships Received}
\begin{tabular}{rl}
\textsc{2020} & Honorable Mention, Graduate Research Fellowship Program, \textbf{National Science Foundation} \\
\textsc{2019} & F. Paul Brady Graduate Fellowship, \UCD \\ %winter quarter of 1st year
\textsc{2014-2018} & Gold Scholar Award, \UMN \\
\textsc{2014-2018} & Dean's List, College of Science and Engineering, \UMN \\
\textsc{2014} & National Merit Scholar, \textbf{National Merit Scholarship Corporation} \\
% \multicolumn{2}{c}{} \\
\end{tabular}


%%%%%%%%%%%%%%%%%%%%%%%%%%%%%%%%%%%%%%%%%%%%%%%%%%
%%% Outreach efforts
%%%%%%%%%%%%%%%%%%%%%%%%%%%%%%%%%%%%%%%%%%%%%%%%%%
\section{Outreach Efforts}
\begin{tabular}{R{1.5cm}|L{14cm}}
\emph{Present} & \textsc{Nuclear Forensics}, \UCD \\
\textsc{Jan 2020} & {\footnotesize Helped to develop an outreach program to provide undergraduate students with an introduction to experimental high energy physics. The program, entitled "Nuclear Forensics: Dusting for the Fingerprints of Radioactivity," seeks to provide untapped groups with a hands-on experience to identify trace radioisotopes through Neutron Activation Analysis.} \\

\end{tabular}

%%%%%%%%%%%%%%%%%%%%%%%%%%%%%%%%%%%%%%%%%%%%%%%%%%
%%% Relevant Skills
%%%%%%%%%%%%%%%%%%%%%%%%%%%%%%%%%%%%%%%%%%%%%%%%%%
\section{Relevant Skills}
\begin{tabular}{rl}
Basic Knowledge: & 
	Machine shop training
	%\textsc{Linux}, 
	%ubuntu, 
	\\
	
Intermediate Knowledge: &   
	Machine learning,
	Bash,
	{\fb \LaTeX}\setmainfont[SmallCapsFont=Fontin-SmallCaps.otf]{Fontin.otf} 
	\\

Advanced Knowledge: & 
	C/C\texttt{++},
	Python,
	\textsc{Geant4} (including optical Monte Carlos),
	ROOT,
	MS Office
	\\
% \multicolumn{2}{c}{} \\
\end{tabular}


\section{Teaching Experience}
\begin{tabular}{R{1.5cm} | L{14cm}}
\textsc{Fall 2018} & Physics 7A, \UCD \\
& {\footnotesize Taught lab and discussion sections for an introductory physics course for non-physics majors.} \\
\end{tabular}



\end{document}
